\documentclass[10pt,titlepage]{article}
\usepackage{graphicx}
%\usepackage{graphics}
\usepackage{epsfig}
\usepackage{amsmath}
\usepackage{amssymb}
\usepackage{amsthm}
\usepackage{booktabs}
\usepackage{stmaryrd}
\usepackage{url}
%\usepackage{longtable}
\usepackage[figuresright]{rotating}

\usepackage{polski}
\usepackage[utf8]{inputenc}
\usepackage[T1]{fontenc}

\usepackage{geometry}
\usepackage{pslatex}
%\usepackage{ulem}

\usepackage{listings}
\usepackage{url}
%\usepackage{Here}

\usepackage{color}
\usepackage{tcolorbox}
\tcbuselibrary{skins}

\colorlet{xlightblue}{blue!5}

\newtcolorbox{beamerlikethm}[1]{
  title=#1,
  beamer,
  colback=xlightblue,
  colframe=blue!30,
  fonttitle=\bfseries,
  left=1mm,
  right=1mm,
  top=1mm,
  bottom=1mm,
  middle=1mm
}

\definecolor{szary}{gray}{0.75}% jasnoszary

%\setlength{\textwidth}{400pt}

\lstset{numbers=left,
			numberstyle=\tiny, 
			basicstyle=\scriptsize\ttfamily, 
			breaklines=true, 
			captionpos=b, 
			tabsize=2}

\usepackage[ruled,vlined,linesnumbered]{algorithm2e}

\vfuzz2pt % Don't report over-full v-boxes if over-edge is small
\hfuzz2pt % Don't report over-full h-boxes if over-edge is small


\newcommand{\RR}{\mathbb{R}}
\newcommand{\NN}{\mathbb{N}}
\newcommand{\QQ}{\mathbb{Q}}
\newcommand{\ZZ}{\mathbb{Z}}
\newcommand{\TAB}{\hspace{0.50cm}}
\newcommand{\IFF}{\leftrightarrow}
\newcommand{\IMP}{\rightarrow}

\newcommand{\PRZYKLAD}[1]{\par \noindent{\color{blue}PRZYKŁAD:}\\ {\color{szary}#1}\par}

\newtheorem{theorem}{Twierdzenie}[section]
\newtheorem{lemma}{Lemat}[section]
\newtheorem{example}{Przykład}[section]
\newtheorem{corollary}{Wniosek}[section]
\newtheorem{definition}{Definicja}[section]

\hyphenation{wszy-stkich ko-lu-mnę każ-da od-leg-łość
   dzie-dzi-ny dzie-dzi-na rów-nych rów-ny
   pole-ga zmie-nna pa-ra-met-rów wzo-rem po-cho-dzi
   o-trzy-ma wte-dy wa-run-ko-wych lo-gicz-nie
   skreś-la-na skreś-la-ną cał-ko-wi-tych wzo-rów po-rzą-dek po-rząd-kiem
   przy-kład pod-zbio-rów po-mię-dzy re-pre-zen-to-wa-ne
   rów-no-waż-ne bi-blio-te-kach wy-pro-wa-dza ma-te-ria-łów
   prze-ka-za-nym skoń-czo-nym mo-żesz na-tu-ral-na cią-gu tab-li-cy
   prze-ka-za-nej}


\begin{document}

\pagestyle{empty} %To jest strona tytułowa, bez numeracji

\begin{titlepage}
\vspace*{\fill}
\begin{center}
\begin{picture}(300,510)
  \put( 10,520){\makebox(0,0)[l]{\large \bf \textsc{Wydział Podstawowych Problemów Techniki}}}
  \put( 10,500){\makebox(0,0)[l]{\large \bf \textsc{Politechniki Wrocławskiej}}}
  \put( 10,315){\makebox(0,0)[l]{\Huge  \bf \textsc{Aplikacja do organizacji}}}
  \put( 10,290){\makebox(0,0)[l]{\Huge  \bf \textsc{imprez okolicznościowych}}}
  \put(100,240){\makebox(0,0)[l]{\large     \textsc{Jarosław Mirek}}}

  \put(170, 80){\makebox(0,0)[l]{\large  {Praca inżynierska napisana}}}
  \put(170, 60){\makebox(0,0)[l]{\large  {pod kierunkiem}}}
  \put(170, 40){\makebox(0,0)[l]{\large  {dr. inż. Marcina Zawady}}}

  \put(100,-80){\makebox(0,0)[bl]{\large \bf \textsc{Wrocław 2014}}}
\end{picture}
\end{center}
\vspace*{\fill}
\end{titlepage}

\tableofcontents

\newpage

\pagestyle{headings}  %Zaczynamy właściwą część dokumentu

\section*{Wstęp}      %* oznacza, że ta sekcja nie będzie numerowana   

Analizując kwestię planowania działań jednostki można dojść do wniosku, że społeczeństwo bardzo sprawnie radzi sobie z planowaniem krótko- oraz długoterminowym.
Nie jest dla nas problemem, aby określić co będziemy robić w nadchodzącym tygodniu lub, że w przeciągu pięciu lat skończymy studia i rozpoczniemy pierwszą pracę.
Kłopoty pojawiają się jednak podczas planowania w średnim terminie, czyli np. miesiąc, rok.

Okazuje się zatem, że w obliczu konieczności organizacji imprezy okolicznościowej (szczególnie formalnej) należy zadbać o wiele szczegółów, które razem mogą stanowić
nie lada wyzwanie planistyczne. Dodatkowo dosyć często jest to wyzwanie, przed którym stajemy właśnie w średnim terminie. Stąd potrzebne jest staranne rozplanowanie 
składowych wydarzenia, aby skutecznie je zorganizować w tym ograniczonym czasie.

Wraz z duchem czasu kartka i długopis odchodzą do lamusa, a my chcemy mieć stały dostęp do ``organizera'' naszego wydarzenia. Dlatego też idealnym rozwiązaniem
wydaje się być przeniesienie standardowego notesu do smartfona, w taki sposób, aby maksymalnie ułatwić kontrolę nad organizowanym wydarzeniem oraz usprawnić jego planowanie.
Przegląd rynku aplikacji smartfonowych pokazuje, że systemy służące organizacji wydarzeń są nastawione na konkretny typ okoliczności np. wesele lub nieformalną imprezę.
Istnieje jednak luka dla aplikacji, która:

\begin{itemize}
 \item posłuży do organizacji różnego typu wydarzeń,
 \item pozwoli organizować zarówno wydarzenia o charakterze bardzo uroczystym, jak i te nieformalne,
 \item zintegruje wszystkich, którzy są uczestnikami wydarzenia,
 \item wyręczy organizatora w części obowiązków.
\end{itemize}


W związku z powyższym, celem zrealizowanej pracy dyplomowej było zaprojektowanie oraz implementacja rozproszonego systemu dedykowanemu systemowi operacyjnemu Android.
System jest w stanie wspomóc organizację imprezy okolicznościowej na kilku płaszczyznach, a wymagania funkcjonalne były następujące:

\begin{itemize}
 \item wspieranie organizacji bardzo uroczystych wydarzeń, jak i tych nieformalnych,
 \item skupienie wokół wydarzenia wszystkich jego uczestników,
 \item możliwość deklaracji uczestników na prezenty lub rzeczy,
 \item skoncentrowanie ważnych elementów imprezy (kontakty, notatki, listy rzeczy do zrobienia) w aplikacji.
\end{itemize}

Jako system o zbliżonej funkcjonalności można przytoczyć serwis Facebook, wraz z oferowaną przezeń funkcjonalnością wydarzeń.
Facebook'owe wydarzenia mają jednak przede wszystkim funkcjonalność informacyjną oraz służą za miejsce do wymiany myśli przez zaproszonych użytkowników.
Zrealizowany projekt rozszerza jednak znacznie ten pomysł, dodając możliwość jednoznacznej deklaracji uczestników, co do kwestii, którymi się zajmą.
Ponadto system oferuje funkcjonalności stricte planistyczne, dzięki którym organizator w formie ``kreatora'', zostanie poprowadzony przez proces pozwalający 
skutecznie zaplanować imprezę okolicznościową.

Niniejsza praca składa się z czterech rozdziałów. W pierwszym rozdziale omawiamy dokładniej problem, oraz sposób interakcji użytkowników ze sobą, a także
przypadki użycia aplikacji.

Rozdział drugi zawiera szczegółowy projekt systemu. Opiszemy w nim poszczególne komponenty składające się na całość zaimplementowanej aplikacji. Przeanalizujemy
aplikację kliencką, stronę serwera, a także bazę danych i komunikację pomiędzy tymi składowymi. Zawrzemy w nim również opis algorytmu generowania unikalnego i losowego
kodu, który jest kluczem dostępu gości do wydarzenia.

Rozdział trzeci uszczegółowi użyte technologie oraz pewne kwestie implementacyjne, którymi cechuje się napisany system.

Czwarty rozdział dotyczyć będzie tematu wdrożenia systemu, natomiast rozdział końcowy podsumuje otrzymane wyniki, związane z nimi możliwe kierunki rozwoju,
a także wypunktuje powodzenie implementacyjne systemu.

\newpage
\section{Analiza problemu}
W tym rozdziale zajmiemy się przede wszystkim tym, co charakteryzuje problem wykonania rozproszonej aplikacji mobilnej dla tematu naszej pracy. Zawrzemy w nim
ponadto sposób podziału ról pomiędzy użytkownikami oraz zmianą struktury aplikacji w zależności od typu organizowanego wydarzenia.

Dla ustalenia uwagi rozdział ten rozpoczniemy od zagadnienia, z którego wyrósł pomysł stworzenia opisywanego w pracy systemu:

\begin{beamerlikethm}{}
Zwyczajowo para, która planuje zawrzeć związek małżeński organizuje przyjęcie weselne. Jednym z nierozerwalnych elementów takiego wydarzenia są prezenty.
Niekiedy młoda para tworzy listę prezentów, które chciałaby otrzymać od gości, a listę taką najczęściej dzierży świadek i to on nadzoruje całe przedsięwzięcie
oraz deklaracje gości dotyczące kupna danego przedmiotu. Nie jest to jednak zbyt wygodna forma, ponieważ goście nie zawsze mają możliwość dokładnego przeanalizowania
listy, a nadzorca listy musi poświęcić swój czas dużej ilości gości. Ponadto mogą pojawić się niedogodności komunikacyjne, które zaowocują duplikatem prezentu. Wady można
jeszcze mnożyć. Dlaczego więc nie pomyśleć o liście prezentów w interaktywnej, cyfrowej formie? Wystarczy dodatkowa karteczka dołączona do zaproszenia, wraz z kodem
oraz instrukcją dotyczącą jego użycia.
\end{beamerlikethm}

Przyjęcie weselne to jednak nie tylko prezenty, ale także mnóstwo obowiązków organizacyjnych, setki telefonów i formalności. Jeśli jednak głębiej się zastanowimy
te same problemy dotyczą organizacji bankietu, urodzin czy nawet nieformalnej uroczystości, które jest współorganizowane przez wszystkich uczestników.

Stąd też jeśli nasza aplikacja ma tym problemom zaradzić, to musimy rozważyć kilka istotnych zagadnień:

\begin{enumerate}
 \item Użytkownicy i ich role:
 \\ Przede wszystkim musimy podzielić użytkowników na organizatorów i uczestników.
 Organizator to użytkownik, który decyduje czy wydarzenie ma charakter oficjalny, czy też nie, ustala listę gości, definiuje listę prezentów (lub, w przypadku
 nieformalnej, współorganizowanej imprezy, przedmioty/dania/napoje, które sa potrzebne), a także powinien móc zdefiniować listę spraw, które należy dopełnić czy też
 posiadać prosty skorowidz, umożliwiający kontakt z firmami zajmującymi się salą, cateringiem czy orkiestrą. Powinien on także posiadać funkcjonalności uczestnika.
 
 Uczestnik jest natomiast użytkownikiem, który może wskutek zalogowania kodem wydarzenia przejrzeć jego szczegóły, tj. gdzie i kiedy się odbędzie czy też listę gości.
 Dodatkowo uczestnik może w formie systemu komentarzy toczyć dyskusję dotyczącą wydarzenia, a także zadeklarować chęć zakupu prezentu z listy.
 
 \item Tajność wydarzenia
 \\ Wydarzenie powinno być przedmiotem zainteresowania uczestniczących w nim gości. Wszelkie szczegóły nie powinny być dostępne każdemu użytkownikowi aplikacji, a jedynie
 osobom, które organizator mianował uczestnikami. Rozwiązaniem, które zostało wymyślone na potrzeby aplikacji jest losowy, unikalny kod, który pozwoli zalogować się do
 wydarzenia. Organizator, tworząc uroczystość w systemie otrzyma stosowny ciąg alfanumeryczny, które może później przekazać zainteresowanym.
 
 
 \item Weryfikacja użytkowników
 \\ System kodu logującego tworzy możliwość, że użytkownik wpisując losowo ciąg cyfr i znaków alfabetu zaloguje się do jakiegoś wydarzenia, którego uczestnikiem być nie powinien.
 Stąd też należy zadbać o weryfikację użytkowników. Z pomocą przychodzi konto w portalu Facebook. W tej chwili bardzo dużo osób posiada takie konto, dzięki czemu można bardzo łatwo
 zweryfikować kim jest zalogowany w aplikacji użytkownik.
 Oprócz tego uniemożliwia to wtargnięcie w kompetencje organizatora.
 
 \item Stworzenie części serwerowej
 \\ System, który w założeniach ma integrować użytkowników o różnych rolach musi posiadać zdalny komponent, który będzie przechowywał wpisywane do systemu informacje
 oraz odpowiednio je propagował pomiędzy nimi. Serwer powinien zatem być zintegrowany z bazą danych oraz umożliwiać prostą komunikację ze smartfonem.
\end{enumerate}


Ważnym elementem systemu z podziałem ról jest również taki projekt aplikacji, który pozwoli na dynamiczną zmianę funkcjonalności w zależności od typu użytkownika. Uczestnik
nie powinien być w stanie dodawać prezentów do listy, a od organizatora nie oczekuje się możliwości deklaracji na prezent, który sam ma otrzymać. Stąd też w naszym systemie
należało stworzyć aplikację, która połączy widoki dla obu ról, oraz serwer, który odpowiednio roześle dane.

Przechodząc do analizy komunikacji pomiędzy aplikacją mobilną, a serwerem należało zastanowić się w jaki sposób przesyłać dane. Dodatkowo należało zaprojektować integrację
serwera z bazą danych tak, aby te trzy komponenty niezawodnie współpracowały, reagowały na nietypowe zdarzenia i były bezpieczne.

Ostatecznym elementem, który w decydujący sposób wpłynąłby na sukces komercyjny systemu jest interfejs użytkownika aplikacji klienckiej. Funkcjonalna aplikacja wymaga, aby
jego interfejs nie wymagał przedzierania się przez gąszcz ekranów, a jego składowe powinny mieć intuicyjny charakter, ew. wsparty elementem pomocy, jaką można uzyskać w odpowiednim
miejscu.

\newpage
\section{Projekt systemu}
W tej części pracy zajmiemy się szczegółowym opisem komponentów odpowiadających za całość systemu. Opiszemy w jaki sposób komponenty te współpracują ze sobą, a także 
poruszymy kwestie związane z komunikacją i algorytmami wymaganymi do sprawnego funkcjonowania systemu.

\subsection{Założenia}
W założeniach system składa się z czterech komponentów: strony internetowej, aplikacji mobilnej oraz serwera zintegrowanego z bazą danych.

\begin{itemize} % to trzeba bedzie przejrzec w zaleznosci od tego jak potoczy sie ze strona
 \item Strona internetowa powinna oferować możliwość integracji posiadanego przez użytkownika konta Facebook z aplikacją kliencką.
 Po zintegrowaniu konta można rozpocząć użytkowanie aplikacji mobilnej. Powinna także zawierać te same funkcjonalności co aplikacja mobilna.
 \item Aplikacja mobilna to element, który pozwoli na pobieranie i przesyłanie danych użytkowników do serwera. Pozwoli również edytować, przetwarzać i umieszczać
dane na ekranie urządzenia. 
 \item Serwer to serce aplikacji. Najważniejszy element systemu, który powinien identyfikować użytkowników, dbać o prawidłową propagację danych, jak również
 sprawnie współpracować ze zintegrowaną bazą danych.
 \item Baza danych, to komponent, w którym przechowywane są wszystkie informacje użytkowników. Ponadto zawiera ona informacje związane z systemem logowania użytkowników do systemu.
\end{itemize}

\subsection{Architektura systemu}
System jest oparty o tzw. model ``klient-serwer''. Oznacza to, że istnieje centralne miejsce w systemie (serwer), które operuje na wszystkich danych, przetwarza je, zapamiętuje, a także w odpowiednim
formacie rozsyła do klientów. W takim modelu element klienta służy przede wszystkim do pobierania i wyświetlania danych, a także do ich wprowadzania do systemu.

napisze tu jeszcze o integracji z baza danych, oraz komunikacja, a takze ze serwer obsluguje i strone i androida. moze cos jeszcze

\subsection{Aplikacja mobilna}

\subsection{Aplikacja internetowa}

\subsection{Projekt serwera}

\subsection{Projekt bazy danych}

\subsection{Opis komunikacji}

\subsection{Opis algorytmów}


\section{Implementacja sytemu}

\subsection{Opis technologii}
\subsubsection{Aplikacja mobilna}
\subsubsection{Aplikacja internetowa}
\subsubsection{Serwer aplikacji}
\subsection{Szczegóły implementacyjne}

\section{Instalacja i wdrożenie systemu}

\section{Dodatek A - Opis płyty}

\section*{Podsumowanie}

%%%%%%%%%%%%%%%%%%%%%%%%%%%%%%%%%%%%%%%%%%%%%%%%%%%%%%%%%%%%%%%%%%%%%%%%%%%%%%
%%%%%%%%%%%%%%%%%%%%%%%%%%%%%%% BIBLIOGRAFIA %%%%%%%%%%%%%%%%%%%%%%%%%%%%%%%%%
%%%%%%%%%%%%%%%%%%%%%%%%%%%%%%%%%%%%%%%%%%%%%%%%%%%%%%%%%%%%%%%%%%%%%%%%%%%%%%

\bibliographystyle{plain}
%\marginpar{Znajdź w internecie porządnie zredagowane cytowania}
\bibliography{P2P}

\end{document}
